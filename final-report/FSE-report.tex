\documentclass{scrartcl}
\usepackage[utf8]{inputenc}
\usepackage{hyperref}
\usepackage{url}
\usepackage{natbib}
\usepackage{graphicx}
\usepackage{tabularx}
\usepackage[italian]{cleveref}
\usepackage[italian]{babel}


\newcommand{\emailaddr}[1]{\href{mailto:#1}{\texttt{#1}}}

\title{\LARGE
    FSE Ontology v2.0
}

\author{
    Alberto Marfoglia \\ \emailaddr{alberto.marfoglia@studio.unibo.it}
    \and 
    Christian D'Errico\\ \emailaddr{christian.derrico@studio.unibo.it} 
}

\date{Giugno 2022}

\begin{document}

\maketitle

\begin{abstract}
    Il Fascicolo Sanitario Elettronico (FSE) è lo strumento che abilita il
    cittadino alla consultazione di tutta la sua storia sanitaria,
    condividendola con medici, infermieri ed eventuali tutori sulla base di
    permessi. Le informazioni raccolte sono strutturate secondo lo standard
    HL7\footnote{Per dettagli: \url{http://www.hl7italia.it/hl7italia_D7/}}, il
    quale oltre a definire la semantica dei documenti scambiati ne dettaglia
    anche le relazioni.
\end{abstract}

\newpage
\tableofcontents
\newpage

\section{Obiettivo}
Il progetto si pone come obiettivo l'estensione di una già esistente ontologia
\cite[]{fse-ontology}, la quale, data la complessità dello standard HL7, non
copre tutti i concetti necessari per la realizzazione del Fascicolo Sanitario.
Le nuove informazioni integrate potranno essere sfruttate per la costruzione di
nuove query SPARQL e la progettazione di nuove schermate lato applicativo
(\textit{frontend}). Un ulteriore requisito non funzionale, sarà quello relativo
alla stesura di un'adeguata documentazione sia per quanto riguarda l'ontologia
FSE di riferimento (non sufficientemente descritta) sia in merito ai nuovi
concetti introdotti, motivando le scelte perseguite.
\subsection{Ontologie esterne}
L'ontologia FSE raccoglie al suo interno concetti derivanti da sorgenti esterne:
\begin{itemize}
    \item FOAF (\textit{Friend Of A Friend})\footnote{FOAF
    dettagli:\url{http://xmlns.com/foaf/spec/}}: impiegata nella definizione del
    paziente, del personale sanitario e delle relazioni che intercorrono tra
    questi;
    \item ORG (\textit{Organization})\footnote{ORG dettagli:
    \url{https://www.w3.org/TR/vocab-org/}}: sfruttata per modellare le varie
    tipologie di strutture sanitarie come ospedali, laboratori, RSA ecc.;
    \item DOID (\textit{Disease Ontology Identifiers})\footnote{DOID dettagli:
    \url{https://www.ebi.ac.uk/ols/ontologies/doid}}: descrive tutte le
    patologie dell'uomo fin'ora conosciute;
    \item DRON (\textit{Drug Ontology})\footnote{DRON dettagli:
    \url{https://bioportal.bioontology.org/ontologies/DRON}}: classifica tutti i
    medicinali presenti nella National Library of Medicine\footnote{Per
    dettagli: \url{https://www.nlm.nih.gov/}}.
    \item LOINC (\textit{Logical Observation Identifier Names and
    Codes})\footnote{LOINC dettagli:
    \url{https://loinc.org/document-ontology/}}: basata sull'omonimo database,
    fornisce identificatori univoci ed universali in riferimento a termini e
    documentazione sanitaria. 
\end{itemize}
Il progetto proposto si integra con le ontologie appena elencate, andandone a
valorizzare il contenuto informativo sfruttabile su più livelli di dettaglio. Un
valido esempio può essere LOINC i cui codici permetto di identificare
univocamente le diverse tipologie di documenti HL7 e le loro sezioni. 
%
Inoltre l'operazione di estensione aggiungerà nuovi concetti provenienti da
altre ontologie, individuate a seguito di una fase di ricerca supportata dal
portale \textit{Ontobee}\footnotetext{Per dettagli:
\url{https://www.ontobee.org/}}:
\begin{itemize}
    \item OMIT (\textit{MicroRNA Target ontology}\footnote{OMIT dettagli:
    \url{https://bioportal.bioontology.org/ontologies/OMIT}}): definisce uno
    standard di scambio dati/elementi nel dominio del microRNA (miR);  
    \item NCIT (\textit{National Cancer Institute Thesaurus}\footnote{NCIT
    dettagli: \url{https://bioportal.bioontology.org/ontologies/NCIT}}):
    vocabolario per l'assistenza clinica, la ricerca traslazionale e di base,
    l'informazione pubblica e le attività amministrative.
\end{itemize}
%
\subsection{Analisi del dominio}
I documenti clinici HL7 si strutturano su due livelli: \texttt{Header} e
\texttt{Body}. Il primo raccoglie meta-dati ed è comune a tutti i
\texttt{ClinicalDocument} (radice della gerarchia di documenti), fatta eccezione
per alcuni concetti non obbligatori e quindi non di interesse per l'analisi. Il
secondo rappresenta il vero e proprio corpo del documento che può essere
descritto in maniera "non strutturata", quindi mediante del testo libero, oppure
"strutturata" cioè avvalendosi del concetto di \textbf{sezione} che diventerà il
fulcro della nuova estensione.
%
\section{Estensioni}
Allo stato attuale, tutti i documenti del FSE sono sottoclassi di
\textit{ClinicalDocument}, da questa ereditano alcuni concetti fondamentali
(Figura \ref{fig:clinicalDocument}) che nello standard HL7 sono raccolti
all'interno dello \texttt{Header}. Di seguito vengono proposti alcuni schemi che
oltre a graficare la modellazione dell'ontologia preesistente, evidenziano (in
rosso) le nuove estensioni.
\subsection*{ClinicalDocument}
Il \texttt{clinicalDocument} si relaziona con diverse entità, per quanto
riguarda quelle "foglia" (\textit{Data Properties}):
\begin{itemize}
    \item \texttt{fse:body}: permette di descrivere agevolmente il documento in
    maniera "non strutturata" (testo libero);
    \item \texttt{fse:createdAt}: registra l'istante di creazione del documento;
    \item \texttt{fse:languageCode}: indica la lingua in cui è redatto il documento,
    nel contesto italiano sarà sempre valorizzato a \texttt{<languageCode
    code="it-IT"/>};
    \item \texttt{fse:realmCode}: indica il dominio di appartenenza del documento,
    nel contesto italiano corrisponde a \texttt{<realmCode code="IT"/>};
    \item \texttt{fse:versionNumber}: permette di marcare le diverse versioni di uno
    stesso documento a seconda della revisione.
\end{itemize}
Altre relazioni degne di nota:
\begin{itemize}
    \item \texttt{fse:inFulfillmentOf}: identifica la richiesta che ha determinato
    la produzione del documento. L'estensione prevede di impiegare come oggetto
    (\texttt{range}) del predicato l'entità \texttt{prescription};
    \item \texttt{fse:hasPreviousVersion}: rappresenta la versione precedente del documento;
    \item \texttt{fse:hasLatestVersion}: traccia l'ultima versione del documento;
    \item \texttt{fse:refersTo}: rappresenta il paziente a cui fa riferimento il documento;
    \item \texttt{fse:hasCode}: indica la tipologia di documento in base alla codifica LOINC;
    \item \texttt{fse:hasBeenSigned}: permette di tracciare l'evento di avvenuta
    firma tramite la classe \texttt{fse:sign} e il medico firmatario
    (\texttt{fse:hasLegalAuthenticator});
    \item \texttt{fse:componentOf}: descrive l'incontro tra l'assistito e la struttura sanitaria;
    \item \texttt{fse:confidentialityLevel}: indica il livello di
    confidenzialità del documento sulla base di un vocabolario (Tabella
    \ref{table:confidentialityCode}) modellato dall'entità
    \texttt{fse:confidentialityCode}. 
\end{itemize}

\begin{table}[h]
    \begin{tabularx}{\textwidth}{|l|X|}
    \hline
    \textbf{Codice}     & \textbf{Descrizione}                                                                                                   \\ \hline
    N (normal)          & Il paziente, i delegati e gli operatori autorizzati possono accedere al documento.                                     \\ \hline
    R (restricted)      & Accesso ristretto soltanto al personale medico o sanitario che ha un mandato di cura attivo in relazione al documento. \\ \hline
    V (very restricted) & Accessibile solo al paziente e dal medico autore del referto.                                                          \\ \hline
    \end{tabularx}
    \caption{Livelli dei permessi dei documenti HL7.}
    \label{table:confidentialityCode}
\end{table}

\begin{figure}
    \includegraphics[width=\textwidth]{images/clinicalDocument.png}
    \caption[]{Estensione del \texttt{ClinicalDocument}}
    \label{fig:clinicalDocument}
\end{figure}

\subsection*{Sezioni}
Un \textit{ClinicalDocument} schematizza informazioni differenti sulla base
della tipologia (RDA, VAC, DE, ecc.) dello stesso. La differenziazione è dettata
dal contenuto del \texttt{Body} e, nello specifico, dalle sezioni che lo
compongono. Si è quindi ritenuto necessario introdurre quest'ultimo concetto
mediante delle relazioni di sottoclasse (Figura \ref*{fig:sections}). Ogni
referto può essere assemblato da più sezioni, di queste, solamente quelle
obbligatorie sono state oggetto di studio. L'entità \texttt{section} oltre ad
amplificare il potenziale semantico dell'ontologia, rende quest'ultima ancora
più aderente allo standard HL7, per esempio aggiungendo una relazione
\texttt{fse:hasCode} che permette di identificare la sezione tramite un codice
LOINC. La relazione \texttt{fse:hasSection}, e la sua inversa
\texttt{fse:isPartOf}, fungono da collante tra il \texttt{clinicalDocument} e le
sue molteplici sezioni.
\begin{figure}
    \includegraphics[width=\textwidth]{images/sections.png}
    \caption[]{Tipologie di referti HL7.}
    \label{fig:sections}
\end{figure}
\newline
\newline
A seguire vengono presentate le tipologie di documenti HL7 necessari per la
realizzazione del FSE secondo le direttive\footnote{Standard documentali:
\url{https://www.fascicolosanitario.gov.it/it/Standard-documentali}} del
Ministero della Salute.
%
\subsubsection*{RAD (Referto di Radiologia)}
Riassume i risultati di tutte le indagini afferenti alla specialità radiologica
(Tabella \ref*{table:radSections}), attestando quanto effettuato per
l’inquadramento diagnostico e terapeutico. Il documento è modellato tramite
l'entità \texttt{fse:radiologyReport} ed è identificato dal codice LOINC
\texttt{"68604-8”}.
\begin{table}
    \begin{tabularx}{\textwidth}{|l|l|l|X|}
    \hline
    \textbf{Sezione}    & \textbf{Classe}  & \textbf{LOINC} & \textbf{Descrizione}   \\ \hline
    Esame eseguito & \texttt{fse:performedExam} & 55111-9      & \small{Descrive l’esame
    radiologico oggetto del referto e deve comprendere sia la data che la modalità
    di esecuzione.} \\ \hline Referto        & \texttt{fse:radReport}     &
    18782-3      & \small{Rappresenta l’elemento centrale e riporta al proprio interno una
    descrizione delle valutazioni del medico.}   \\ \hline
    \end{tabularx}
    \caption{Sezioni obbligatorie del documento RAD.}
    \label{table:radSections}
\end{table}
    %
\subsubsection*{LDO (Lettera di Dimissione Ospedaliera)}
La lettera di  dimissione è un documento che viene rilasciato al paziente al
termine di una fase di ricovero ospedaliero e contiene le indicazioni per gli
eventuali controlli o terapie da effettuare (Tabella \ref*{table:ldoSections}).
Rappresentato da \texttt{fse:hospitalDischargeLetter} e univocamente
identificato dal codice LOINC \texttt{"34105-7"}.
\begin{table}
    \begin{tabularx}{\textwidth}{|l|l|l|X|}
    \hline
    \textbf{Sezione}    & \textbf{Classe}  & \textbf{LOINC} & \textbf{Descrizione}   \\ \hline
    Dimissione   & \small{\texttt{fse:dischargeDiagnosis}}  &  11535-2 & \small{Descrive l'elenco
    delle diagnosi di dimissione, in ordine di rilevanza.}      \\ \hline 
    Motivo del ricovero   & \small{\texttt{fse:reasonForAdmission}}     &    46241-6   & \small{Descrive la causa principale che ha determinato il ricovero del paziente
    attraverso la diagnosi di accettazione.}      \\ \hline
    Decorso ospedaliero   & \texttt{fse:progression}     &    8648-8   & \small{Descrivere l’andamento del ricovero, il percorso diagnostico,
    terapeutico, riabilitativo o assistenziale.}      \\ \hline
    \end{tabularx}
    \caption{Sezioni obbligatorie del documento LDO}
    \label{table:ldoSections}
\end{table}
%
\subsubsection*{DE (Documento di Esenzione)}
Descrive la tipologia di esenzioni posseduta dall’assistito che può essere
totale o parziale (Tabella \ref*{table:deSections}). Definito dalla classe
\texttt{fse:exemptionReport}, è associato al codice LOINC \texttt{"57827-8"}.
\begin{table}
    \begin{tabularx}{\textwidth}{|l|l|l|X|}
    \hline
    \textbf{Sezione}    & \textbf{Classe}  & \textbf{LOINC} & \textbf{Descrizione}   \\ \hline
        Esenzione & \texttt{fse:exemption}  & 57827-8 & \small{Contiene le
        informazioni sulle esenzioni.}      \\ \hline 
    \end{tabularx}
    \caption{Sezioni obbligatorie del documento DE}
    \label{table:deSections}
\end{table}
%
\subsubsection*{RSA (Referto di Specialistica Ambulatoriale)} 
RSA è finalizzato a definire uno standard per la refertazione di prestazioni
ambulatoriali specialistiche (visite mediche ed esami strumentali) che non
ricadano nella sfera della medicina di laboratorio, della radiologia ed imaging,
dell'anatomia patologica (Tabella \ref*{table:rsaSections}). Modellato tramite
\texttt{fse:outpatientSpecialistReport} e identificato dal codice LOINC
\texttt{"11488-4"}.
\begin{table}
    \begin{tabularx}{\textwidth}{|l|l|l|X|}
    \hline
    \textbf{Sezione}    & \textbf{Classe}  & \textbf{LOINC} & \textbf{Descrizione}   \\ \hline
    Prestazioni  & \texttt{fse:services}  & 62387-6 & \small{Riporta le prestazioni
    amministrative erogate, e, ove applicabile, le procedure operative e
    cliniche eseguite (es. somministrazione farmaci).}      \\ \hline 
    Referto  & \texttt{fse:osReport}  & 47045-0 & \small{Descrizione delle
    valutazioni del medico e dell’esito della prestazione.}      \\ \hline
    \end{tabularx}
    \caption{Sezioni obbligatorie del documento RSA}
    \label{table:rsaSections}
\end{table}
%
\subsubsection*{VPS (Verbale di Pronto Soccorso)}
VPS riassume i risultati di tutte le indagini eseguite in regime di urgenza in
Pronto Soccorso, attestando quanto effettuato per l’inquadramento diagnostico e
terapeutico (Tabella \ref*{table:vpsSections}). Rappresentato da
\texttt{fse:firstAidReport} e associato al codice LOINC \texttt{"59258-4"}.
\begin{table}
    \begin{tabularx}{\textwidth}{|l|l|l|X|}
    \hline
    \textbf{Sezione}    & \textbf{Classe}  & \textbf{LOINC} & \textbf{Descrizione}   \\ \hline
    Motivo della visita  & \texttt{fse:reasonForVisit}  & 46239-0 & \small{Descrive il motivo per cui il paziente accede al Pronto Soccorso, ed il
    problema, il sintomo principale riscontrato o percepito dal paziente.}      \\ \hline 
    Dimissione  & \texttt{fse:discharge}  & 18842-5 & \small{Descrive i dati
    relativi alla fase di dimissione, tra cui la diagnosi di dimissione, la
    prognosi, l’esito del trattamento.}      \\ \hline
    Triage & \texttt{fse:triage}  & 54094-8 & \small{Definisce la fase di triage
    dell’accesso in Pronto Soccorso.}      \\ \hline 
    Modalità di trasporto  & \texttt{fse:modeOfTransport}  & 11459-5 & \small{Specifica la modalità di trasporto (modalità arrivo) del paziente al
    Pronto Soccorso ed il responsabile dell’invio al Pronto Soccorso.}      \\ \hline
    \end{tabularx}
    \caption{Sezioni obbligatorie del documento VPS}
    \label{table:vpsSections}
\end{table}
    %
\subsubsection*{VAC (Vaccinazioni)}
VAC è il documento che attesta l’avvenuta somministrazione della/le
vaccinazione/i somministrate all’assistito in una certa data (Tabella
\ref*{table:vacSections}). Definito tramite \texttt{fse:immunization} e
identificato da codice LOINC \texttt{"87273-9"}.
\begin{table}
    \begin{tabularx}{\textwidth}{|l|l|l|X|}
    \hline
    \textbf{Sezione}    & \textbf{Classe}  & \textbf{LOINC} & \textbf{Descrizione}   \\ \hline
    \small{Scheda della singola Vacc.}  & \texttt{fse:immunizationCard}  & 11369-6 & \small{Descrive i dettagli della somministrazione}      \\ \hline 
    \end{tabularx}
    \caption{Sezioni obbligatorie del documento VAC}
    \label{table:vacSections}
\end{table}
%
\subsubsection*{RML (Referto di Medicina di Laboratorio)}
RML rappresenta il risultato di una prestazione svolta da uno specifico
laboratorio interno ad una struttura sanitaria. Rappresentato da
\texttt{fse:laboratoryMedicineReport} con codice LOINC \texttt{"11502-2"}.
L'identificativo LOINC della sezione (Tabella \ref*{table:rmlSections}) cambia
in base alla tipologia di esame.
\begin{table}
    \begin{tabularx}{\textwidth}{|l|l|c|X|}
    \hline
    \textbf{Sezione}    & \textbf{Classe}  & \textbf{LOINC} & \textbf{Descrizione}   \\ \hline
    Specialità  & \texttt{fse:specialty}  & * & \small{Contiene i risultati di esami di laboratorio afferenti ad una singola specialità (ad esempio uno studio microbiologico) o a più specialità}      \\ \hline 
    \end{tabularx}
    \caption{Sezioni obbligatorie del documento RML}
    \label{table:rmlSections}
\end{table}
% 
\subsubsection*{PSS (Profilo Sanitario Sintetico)}
PSS è il documento socio-sanitario informatico redatto e aggiornato dal MMG/PLS,
che riassume la storia clinica dell’assistito e la sua situazione corrente
conosciuta (Tabella \ref*{table:pssSections}). In particolare, lo scopo del
documento è quello di favorire la continuità di cura. Definito dalla classe
\texttt{fse:summaryHealthProfile} con codice LOINC \texttt{"60591-5"}.
\begin{table}
    \begin{tabularx}{\textwidth}{|l|l|l|X|}
    \hline
    \textbf{Sezione}    & \textbf{Classe}  & \textbf{LOINC} & \textbf{Descrizione}   \\ \hline
    Allergie  & \texttt{fse:alerts}  & 48765-2 & \small{Raccoglie ogni
    informazione relativa ad allergie, reazioni avverse, ed allarmi passati
    o presenti inerenti il paziente, se ritenute rilevanti.}      \\ \hline 
    Terapie  & \texttt{fse:medications}  & 10160-0 & \small{Deputata alla
    registrazione di tutte le informazioni inerenti le terapie
    farmacologiche (prescrizioni, somministrazioni, ecc.)}      \\ \hline
    Lista dei Problemi	  & \small{\texttt{fse:historyOfDiseases}}  & 11450-4 & \small{Documenta tutti i problemi clinici rilevanti noti al momento in
        cui è stato generato il documento (problemi clinici, condizioni,
        sospetti diagnostici e diagnosi certe, sintomi attuali o passati).}      \\ \hline 
    Gravidanze e parto  & \small{\texttt{fse:historyOfPregnancies}}  & 10162-6 &
        \small{Include tutte le informazioni inerenti gravidanze (incluso aborti spontanei), parti, eventuali
        complicanze derivate e stato mestruale}
        \\ \hline
    Vaccinazioni  & \small{\texttt{fse:immunizations}}  & 11369-6 &
    \small{Riporta le informazioni relative allo stato attuale di
    immunizzazione (vaccinazioni) del paziente e alle vaccinazioni
    effettuate}      \\ \hline
    \end{tabularx}
    \caption{Sezioni obbligatorie del documento PSS}
    \label{table:pssSections}
\end{table}
%
\subsection{Certificato Vaccinale}
Attraverso la sezione \texttt{fse:immunizationCard} sarà possibile ripercorrere
le vaccinazioni effettuate negli anni, con conseguente produzione del
Certificato Vaccinale. Si rende necessaria l'introduzione dell'entità
\texttt{fse:substanceAdministration} (Figura \ref*{fig:immunizations}) al fine
di dettagliare l'atto di somministrazione mediante:
\begin{itemize}
    \item \texttt{fse:effectiveTime}: istante di avvenuta vaccinazione;
    \item \texttt{fse:approachSiteCode}: parte anatomica interessata dalla
    somministrazione, il valore è specificato dallo standard HL7 sulla base di
    un vocabolario\footnote{v3-ActSite dettagli:
    \url{https://terminology.hl7.org/3.1.0/CodeSystem-v3-ActSite.html}};
    \item \texttt{fse:doseQuantity}: dose del farmaco inoculato.
\end{itemize}
Da questa si propagano varie relazioni:
\begin{itemize}
    \item \texttt{fse:via}: determina la via di somministrazione mediante
    l'impiego della gerarchia esterna \textbf{OMIT\_0005589};
    \item \texttt{fse:prevents}: specifica la malattia che è in grado di prevenire;
    \item \texttt{fse:consumable}: indica il farmaco impiegato;
    \item \texttt{fse:participant}: definisce la struttura sanitaria in cui si è
    svolta la somministrazione;
    \item \texttt{fse:booster}: tiene traccia degli eventuali richiami,
    identificati da un codice LOINC;
    \item \texttt{fse:includesAdministration}: lega la sezione alla somministrazione.
\end{itemize}

\begin{figure}
    \includegraphics[width=\textwidth]{images/immunizations.png}
    \caption[]{Estensione dell'entità \texttt{fse:immunization} (referto VAC).}
    \label{fig:immunizations}
\end{figure}
%
\subsection{PSS}
L'introduzione della relazione \texttt{fse:hasSection}, nel caso del
\texttt{summaryHealthProfile}, abilita alla raccolta di nuove informazioni
raggruppate per competenza (Figura \ref*{fig:pss}):
\begin{itemize}
    \item \texttt{fse:alerts}: sintetizza tutti i dati relativi ad eventuali
    allergie del paziente. Tramite \texttt{fse:includesAllergy} si referenzia la
    descrizione completa dell'allergia di cui il paziente è affetto. I sintomi
    provocati dall'allergia possono essere rappresentati avvalendosi nuovamente
    della ontologia esterna \textbf{DOID\_1205}.
    \item \texttt{fse:historyOfDiseases}: storico delle eventuali patologie che
    affliggono il paziente. Le patologie sono modellate dall'ontologia
    \textbf{DOID\_4} messa in relazione tramite \texttt{fse:includesDisease}. 
    \item \texttt{fse:medications}: la sezione raccoglie, tramite la
    relazione \texttt{fse:includesTherapy}, tutte le terapie assegnate, e di ognuna
    anche la sostanza somministrata.
    \item  \texttt{fse:historyOfPregnancies}: corrisponde all'elenco delle
    gravidanze associate tramite la relazione \texttt{fse:includesPregnancy}, la
    quale include l'esito finale e il periodo.
\end{itemize}
\begin{figure}
    \includegraphics[width=\textwidth]{images/pss.png}
    \caption[]{Estensione dell'entità \texttt{fse:summaryHealthProfile} (referto PSS).}
    \label{fig:pss}
\end{figure}

\subsection{VPS e RSA}
Seguendo la logica delle precedenti estensioni anche per i referti VPS e RSA, vi
sono una serie di nuove relazioni dettate dall'integrazione di specifiche
sezioni (Figura \ref*{fig:vps-rsa}). Le sezioni dei due documenti hanno un
concetto comune, e cioè quello di \texttt{operatingProcedure}, il quale può
presentarsi in varie forme:
\begin{itemize}
    \item \texttt{fse:procedure}: indica procedure diagnostiche invasive,
    interventistiche, chirurgiche, terapeutiche non farmacologiche;
    \item \texttt{fse:substanceAdministration}: indica procedure di
    somministrazioni farmaceutiche come terapie, vaccinazioni e sedazioni;
    \item \texttt{fse:observation}: indica osservazioni eseguite e parametri
    clinici rilevati durante la prestazione;    
\end{itemize}
\begin{figure}
    \includegraphics[width=\textwidth]{images/vps-rsa.png}
    \caption[]{Estensione dei documenti VPS e RSA.}
    \label{fig:vps-rsa}
\end{figure}

\subsection{RML}
Attraverso l'impiego della sezione \texttt{fse:specialty}, il documento RML è in
grado di raccogliere gli esami effettuati (es. studio microbiologico) ed i
relativi risultati grazie alla relazione \texttt{fse:hasResult} (Figura
\ref*{fig:rml}). Infine è possibile ripercorrere la catena di esami eseguiti,
similmente a quanto fatto per \texttt{fse:clinicalDocument}, mediante le
relazioni \texttt{fse:hasPreviousVersion} e \texttt{fse:hasLatestVersion}.
\begin{figure}[]
    \centering
    \includegraphics[width=0.4\textwidth]{images/rml.png}
    \caption[]{Estensione del documento RML.}
    \label{fig:rml}
\end{figure}

\subsection{Prescrizioni}
Il documento di Prescrizione è un \texttt{clinicalDocument} che non fa
direttamente parte del FSE ma è un'entità di supporto che permette di tracciare
l'ordine che ha dato vita alla creazione di uno specifico referto. Questo lo si
evince anche dalla estensione della relazione \texttt{fse:inFulfillmentOf}
(Figura \ref*{fig:prescription}). Il documento di Prescrizione può
concretizzarsi in diverse classi, ognuna è sottoclasse dell'entità
\texttt{fse:prescription} ed è identificata da uno specifico codice LOINC
(Tabella \ref*{table:prescriptions}).
%
Alle prescrizioni viene associato un livello di priorità preso da un set di
valori predefiniti (Tabella \ref*{table:priorityLevels}). Questo è modellato per
mezzo della classe \texttt{fse:priorityLevel} e la relazione
\texttt{fse:hasPriority}.
%
\begin{table}
    \begin{tabularx}{\textwidth}{|l|l|l|X|}
    \hline
    \textbf{Sezione}    & \textbf{Classe}  & \textbf{LOINC} & \textbf{Descrizione}   \\ \hline
    P. Farmaceutica  & \texttt{fse:pharma}  & 57833-6 & \small{Raccoglie
    informazioni relative ai farmaci prescritti ed al relativo numero di pezzi/confezioni.}      \\ \hline 
    P. Specialistica  & \texttt{fse:specialist}  & 57832-8 & \small{Contiene
    informazioni relative alle prestazioni specialistiche o riabilitative
    prescritte.}      \\ \hline
    P. Ricovero	  & \small{\texttt{fse:admission}}  &  57830-2 &
    \small{Contiene informazioni relative al ricovero prescritto o alla
    richiesta di trasporto, alla priorità richiesta dal prescrittore, ecc.}
    \\ \hline 
    P. Trasporto  & \small{\texttt{fse:transportRequest}}  &  57834-4 &
        \small{Include tutte le informazioni inerenti gravidanze (incluso aborti spontanei), parti, eventuali
        complicanze derivate e stato mestruale}
        \\ \hline
    \end{tabularx}
    \caption{Famiglia delle prescrizioni.}
    \label{table:prescriptions}
\end{table}
%
\begin{figure}[]
    \centering
    \includegraphics[width=\textwidth]{images/prescription.png}
    \caption[]{Schema dell'entità \texttt{fse:prescription}.}
    \label{fig:prescription}
\end{figure}
%
\begin{table}
    \begin{tabularx}{\textwidth}{|c|X|}
    \hline
    \textbf{Codice}    & \textbf{Descrizione}   \\ \hline
    S  & Urgente - nel più breve tempo possibile o, se differibile, entro 72 ore \\ \hline
    A  & Breve entro 10 gg \\ \hline
    EL & Differita - entro 30 gg per le visite, entro 60 gg per gli accertamenti specialistici \\ \hline
    R  & Programmata - senza priorità \\ \hline
    \end{tabularx}
    \caption{Livelli di priorità relativi alle prescrizioni.}
    \label{table:priorityLevels}
\end{table}

\section{Scelte implementative}
La modellazione dei concetti dell'ontologia si pone a metà tra la terminologia
HL7 e quella definita ad-hoc, al fine di mantenere un buon bilanciamento tra
aderenza allo standard e facilità di fruizione (semplice da comprendere e
utilizzare). Nelle sezioni a seguire vengono fornite le scelte implementative e
le relative motivazioni.
\subsection{Gerarchia ClinicalDocument}
Ogni tipologia di documento eredita dalla super-classe la restrizione
\texttt{\underline{only}} sulle relazioni \texttt{fse:hasPreviousVersion} e
\texttt{fse:hasLatestVersion} (Figura \ref*{fig:clinicalDocument_restrictions}),
poiché il dominio e il \textit{range} di queste possono essere uno tra
\texttt{fse:clinicaDocument} che \texttt{fse:test}. Inoltre, per garantire la
consistenza delle informazioni, sono stati specificati altri due vincoli:
\begin{itemize}
    \item \texttt{hasCode \underline{value} L}: ogni tipologia di report è
    identificata univocamente da un codice LOINC (\texttt{L}). A tal proposito,
    come per le sezioni, si è posto un vincolo sull'istanza di quest'ultimo; nel
    caso di \texttt{fse:hospitalDischargeLetter} la proprietà funzionale
    \texttt{fse:hasCode} può assumere solo il valore \texttt{34105-7}. 
    \item \texttt{hasSection \underline{only} (A \underline{or} B ...)}:
    similmente al caso precedente, anche per il predicato
    \texttt{fse:hasSection} sono state introdotte delle limitazioni sul
    \textit{range} così da rispettare la semantica HL7. Nel caso di
    \texttt{fse:hospitalDischargeLetter}, le uniche sezioni valide per la
    composizione del corpo del documento sono \texttt{fse:discharge\newline Diagnosis},
    \texttt{fse:progression} e \texttt{fse:reasonForAdmission}.
\end{itemize}
\begin{figure}[]
    \centering
    \includegraphics[width=\textwidth]{images/clinicalDocument_restrictions.png}
    \caption[]{Restrizioni sulla gerarchia \texttt{fse:clinicalDocument}.}
    \label{fig:clinicalDocument_restrictions}
\end{figure}
Infine ciascuna entità è stata dotata di una descrizione (\texttt{comment}),
avvalendosi della finestra \textit{Annotations}, così da produrre una
documentazione di supporto.
\subsection{Classi enumerate}
Per la modellazione dei concetti \texttt{fse:confidentialityCode} e
\texttt{priorityLevel} ci si è avvalsi di classi enumerate\footnote{Per
dettagli:
\href{https://www.michaeldebellis.com/post/owl-best-practices-using-enumerated-classes-to-define-ordered-values}{https://www.michaeldebellis.com/post/enumerated-classes}}.
Questa scelta nasce dal fatto che le entità interessate possono assumere solo
valori predefiniti, infatti le loro istanze sono state vincolate a specifici
\textit{Individuals}. Inoltre la definizione dei concetti tramite classe
permette anche di esprimere delle proprietà transitive, come
\texttt{fse:isMoreUrgentThan} e \texttt{fse:isLessUrgentThan} (Figura
\ref*{fig:priorityLevel}), che determinano l'ordinamento delle diverse istanze,
in parte inferito dal \textit{reasoner}.
\begin{figure}[]
    \centering
    \includegraphics[width=\textwidth]{images/priorityLevel.png}
    \caption[]{Modellazione della \textit{enumerated class} \texttt{fse:priorityLevel}.}
    \label{fig:priorityLevel}
\end{figure}
\subsection{Gerarchia Observation}
Le procedure sanitarie effettuate possono essere di diversi tipi, uno di questi
e il concetto di osservazione (\texttt{fse:observation}), siccome questo è a sua
volta descritto da diverse entità che raccolgono un'elevata mole di dati, si è
preferito referenziare le classi \texttt{Client Observation} e \texttt{Patient
Observation} dell'ontologia esterna NCIT (Figura \ref*{fig:ncit}).
\begin{figure}[h]
    \centering
    \includegraphics[width=\textwidth]{images/NCIT.png}
    \caption[]{Integrazione dell'ontologia esterna NCIT.}
    \label{fig:ncit}
\end{figure}
\subsection{Semplificazione delle ontologie esterne}
Alcune ontologie esterne come LOINC e OMIT, includono un numero elevato di
classi e assiomi di difficile gestione all'interno di \textit{Protégé}. Per
questo motivo si è deciso di estrapolare da queste i soli concetti necessari per
l'implementazione del FSE. Questa attività a prodotto nuove ontologie di
supporto che al bisogno possono essere sostituite da quelle complete. Nel caso
di LOINC sono stati importati tutti gli \textit{Individuals} e le classi di
contorno che modellano i codici dei \texttt{fse:clinicalDocument} e delle
\texttt{fse:section}. Per quanto riguarda OMIT, le entità di interesse erano
solo quelle relative alle possibili vie di somministrazione. A tal proposito si
è realizzato un "taglio" dell'ontologia (OMIT\_0005589) che comprende uno
specifico gruppo di classi (Figura \ref{fig:omit}).
\begin{figure}[h]
    \centering
    \includegraphics[width=0.55\textwidth]{images/omit.png}
    \caption[]{Integrazione dell'ontologia esterna OMIT.}
    \label{fig:omit}
\end{figure}
%
\subsection{OWL - Property Chains}
Nella precedente versione dell'ontologia vi sono due gerarchie di \textit{Object
Property} (Figura \ref*{fig:chain}) che semplificano la fruizione del grafo
della conoscenza. Nel dettaglio, le proprietà derivanti da
\texttt{fse:hasProfileField}, che ha come dominio \texttt{fse:patient}, vengono
automaticamente inferite sulla base della controparte
\texttt{fse:includesFields}, il cui dominio è \texttt{fse:section}. Il tutto si
rende possibile grazie alle \textit{Object Properties}, funzionalità di OWL 2.  
\begin{figure}[h]
    \centering
    \includegraphics[width=0.8\textwidth]{images/property-chain.png}
    \caption[]{\textit{Object Properties} interessate dalla catena OWL.}
    \label{fig:chain}
\end{figure}
\section{SPARQL}
\section{Applicativo}
\section{Conclusioni}
%
%
%
%
\newpage
\nocite{*} % Includes all references from the `references.bib` file
\bibliographystyle{plain}
\bibliography{references}

\end{document}