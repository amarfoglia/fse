\documentclass{scrartcl}
\usepackage[utf8]{inputenc}
\usepackage{hyperref}
\usepackage{url}
\usepackage{natbib}
\usepackage{graphicx}
\usepackage{tabularx}


\newcommand{\emailaddr}[1]{\href{mailto:#1}{\texttt{#1}}}

\title{\LARGE
    FSE Proposal
}

\author{
    Alberto Marfoglia \\ \emailaddr{alberto.marfoglia@studio.unibo.it}
    \and 
    Christian D'Errico\\ \emailaddr{christian.derrico@studio.unibo.it} 
}

\date{Giugno 2022}

\begin{document}

\maketitle

\begin{abstract}
    Il Fascicolo Sanitario Elettronico (FSE) è lo strumento che abilita il
    cittadino alla consultazione di tutta la sua storia sanitaria,
    condividendola con medici, infermieri ed eventuali tutori sulla base di
    permessi. Le informazioni raccolte sono strutturate secondo lo standard
    HL7\footnote{Per dettagli: \url{http://www.hl7italia.it/hl7italia_D7/}},
    il quale oltre a definire la semantica dei documenti scambiati ne dettaglia
    anche le relazioni.
\end{abstract}

\section{Obiettivo}
Il progetto si pone come obiettivo l'estensione di una già esistente ontologia
\cite[]{fse-ontology}, la quale, data la complessità dello standard HL7, non
copre tutti i concetti necessari per la realizzazione del Fascicolo Sanitario.
Le nuove informazioni integrate potranno essere sfruttate per la costruzione di
nuove query SPARQL e la progettazione di nuove schermate lato applicativo
(\textit{frontend}). Un ulteriore requisito non funzionale, sarà quello relativo
alla stesura di un'adeguata documentazione sia per quanto riguarda l'ontologia
FSE di riferimento (non sufficientemente descritta) sia in merito ai nuovi
concetti introdotti.
\subsection{Ontologie esterne}
L'ontologia FSE raccoglie al suo interno concetti derivanti da sorgenti esterne:
\begin{itemize}
    \item FOAF (\textit{Friend Of A Friend}): impiegata nella definizione del
    paziente, del personale sanitario e delle relazioni che intercorrono tra
    questi;
    \item ORG (\textit{Organization}): sfruttata per modellare le varie
    tipologie di strutture sanitarie come ospedali, laboratori, RSA ecc.;
    \item DOID (\textit{Disease Ontology Identifiers}): descrive tutte le
    patologie dell'uomo fin'ora conosciute;
    \item DrOn (\textit{Drug Ontology}): classifica tutti i medicinali presenti nella National
    Library of Medicine\footnote{Per dettagli: \url{https://www.nlm.nih.gov/}}.
    \item LOINC (\textit{Logical Observation Identifier Names and Codes}):
    basata sull'omonimo database, fornisce identificatori univoci ed universali
    in riferimento a termini e documentazione sanitaria. 
\end{itemize}
In base allo stato attuale del progetto FSE, le ontologie sopra citate possono
essere integrate ulteriormente come per esempio LOINC, che oltre a identificare i
documenti del Fascicolo Sanitario permette di determinare univocamente anche
le loro diverse tipologie di sezioni. 
%
Oltre a queste è possibile introdurre nuove ontologie a supporto del dominio:
\begin{itemize}
    \item UNIT (unità di misura): promuove la semantica relativa alla frequenza e al
    periodo delle terapie oltre ad approfondire la descrizione del dosaggio delle
    somministrazioni.
\end{itemize}

\subsection{Analisi del dominio}
I documenti clinici HL7 si strutturano su due livelli: \texttt{Header} e
\texttt{Body}. Il primo raccoglie meta dati ed è comune a tutti i
\texttt{ClinicalDocument} (radice della gerarchia di documenti), fatta eccezione
per alcuni concetti non obbligatori e quindi non di interesse per l'analisi. Il
secondo rappresenta il vero e proprio corpo del documento che può essere
descritto in maniera "non strutturata", quindi mediante del testo libero, oppure
"strutturata" cioè avvalendosi del concetto di \textbf{sezione} che diventerà il
fulcro della nuova estensione.
%
\section{Estensioni}
Allo stato attuale, tutti i documenti del FSE sono sottoclassi di
\textit{ClinicalDocument}, da questa ereditano alcuni concetti fondamentali
(Figura \ref{fig:clinicalDocument}) che nello standard HL7 sono raccolti
all'interno dello \texttt{Header}. Di seguito vengono proposti alcuni schemi che
oltre a graficare la modellazione dell'ontologia preesistente, evidenziano (in
rosso) le nuove estensioni.
\subsection*{ClinicalDocument}
Il \texttt{clinicalDocument} si relaziona con diverse entità, per quanto riguarda quelle "foglia":
\begin{itemize}
    \item \texttt{body}: permette di descrivere agevolmente il documento in maniera "non strutturata";
    \item \texttt{confidentialityCode}: indica il livello di confidenzialità del
    documento sulla base di un vocabolario (Tabella
    \ref{table:confidentialityCode});
    \item \texttt{createdAt}: registra l'istante di creazione del documento;
    \item \texttt{languageCode}: indica la lingua in cui è redatto il documento,
    nel contesto italiano sarà sempre valorizzato a \texttt{<languageCode
    code="it-IT"/>};
    \item \texttt{realmCode}: indica il dominio di appartenenza del documento,
    nel contesto italiano corrisponde a \texttt{<realmCode code="IT"/>};
    \item \texttt{versionNumber}: permette di marcare le diverse versioni di uno
    stesso documento a seconda della revisione.
\end{itemize}
Altre relazioni degne di nota:
\begin{itemize}
    \item \texttt{inFulfillmentOf}: identifica la richiesta che ha determinato
    la produzione del documento. L'estensione prevede di aggiungere come oggetto
    (\texttt{range}) del predicato l'entità \texttt{prescription};
    \item \texttt{hasPreviousVersion}: rappresenta la versione precedente del documento;
    \item \texttt{hasLatestVersion}: traccia l'ultima versione del documento;
    \item \texttt{refersTo}: rappresenta il paziente a cui fa riferimento il documento;
    \item \texttt{hasCode}: indica la tipologia di documento in base alla codifica LOINC;
    \item \texttt{hasSignatory}: riporta il firmatario del documento (\texttt{hasSupervisor}) e l'istante in cui è avvenuto l'atto;
    \item \texttt{componentOf}: descrive l'incontro tra l'assistito e la struttura sanitaria.
\end{itemize}

\begin{table}[h]
    \begin{tabularx}{\textwidth}{|l|X|}
    \hline
    \textbf{Codice}     & \textbf{Descrizione}                                                                                                   \\ \hline
    N (normal)          & Il paziente, i delegati e gli operatori autorizzati possono accedere al documento.                                     \\ \hline
    R (restricted)      & Accesso ristretto soltanto al personale medico o sanitario che ha un mandato di cura attivo in relazione al documento. \\ \hline
    V (very restricted) & Accessibile solo al paziente e dal medico autore del referto.                                                          \\ \hline
    \end{tabularx}
    \caption{Livelli dei permessi dei documenti HL7.}
    \label{table:confidentialityCode}
\end{table}

\begin{figure}
    \includegraphics[width=\textwidth]{images/clinicalDocument.png}
    \caption[]{Estensione del \texttt{ClinicalDocument}}
    \label{fig:clinicalDocument}
\end{figure}

\subsection*{Sezioni}
Un \textit{ClinicalDocument} schematizza informazioni differenti sulla base
della tipologia (RDA, VAC, DE, ecc.) dello stesso. La differenziazione è dettata
dal contenuto del \texttt{Body} e, nello specifico, dalle sezioni che lo
compongono. Si è quindi ritenuto necessario introdurre quest'ultimo concetto
mediante delle relazioni di sottoclasse (Figura \ref*{fig:sections}). Ogni
referto può essere assemblato da più sezioni, di queste, solamente quelle
obbligatorie sono state oggetto di studio. L'entità \texttt{section} oltre ad
amplificare il potenziale semantico dell'ontologia, rende quest'ultima ancora
più aderente allo standard HL7, per esempio aggiungendo una relazione
\texttt{hasCode} che permette di identificare la sezione tramite un codice
LOINC. La relazione \texttt{hasSection} funge da collante tra il
\texttt{clinicalDocument} e le sue molteplici sezioni.
\begin{figure}
    \includegraphics[width=\textwidth]{images/sections.png}
    \caption[]{Tipologie di referti HL7.}
    \label{fig:sections}
\end{figure}
\newline
\newline
A seguire vengono presentate le tipologie di documenti HL7 necessari per la
realizzazione del FSE secondo le direttive\footnote{Standard documentali:
\url{https://www.fascicolosanitario.gov.it/it/Standard-documentali}} del
Ministero della Salute.
%
\subsubsection*{RAD (Referto di Radiologia)}
Riassume i risultati di tutte le indagini afferenti alla specialità radiologica
(Tabella \ref*{table:radSections}), attestando quanto effettuato per
l’inquadramento diagnostico e terapeutico. Il documento è modellato tramite
l'entità \texttt{fse:radiologyReport} ed è identificato dal codice LOINC
\texttt{"68604-8”}.
\begin{table}
    \begin{tabularx}{\textwidth}{|l|l|l|X|}
    \hline
    \textbf{Sezione}    & \textbf{Classe}  & \textbf{LOINC} & \textbf{Descrizione}   \\ \hline
    Esame eseguito & \texttt{fse:performedExam} & 55111-9      & \small{Descrive l’esame
    radiologico oggetto del referto e deve comprendere sia la data che la modalità
    di esecuzione.} \\ \hline Referto        & \texttt{fse:radReport}     &
    18782-3      & \small{Rappresenta l’elemento centrale e riporta al proprio interno una
    descrizione delle valutazioni del medico.}   \\ \hline
    \end{tabularx}
    \caption{Sezioni obbligatorie del documento RAD.}
    \label{table:radSections}
\end{table}
    %
\subsubsection*{LDO (Lettera di Dimissione Ospedaliera)}
La lettera di  dimissione è un documento che viene rilasciato al paziente al
termine di una fase di ricovero ospedaliero e contiene le indicazioni per gli
eventuali controlli o terapie da effettuare (Tabella \ref*{table:ldoSections}).
Rappresentato da \texttt{fse:hospitalDischargeLetter} e univocamente
identificato dal codice LOINC \texttt{"34105-7"}.
\begin{table}
    \begin{tabularx}{\textwidth}{|l|l|l|X|}
    \hline
    \textbf{Sezione}    & \textbf{Classe}  & \textbf{LOINC} & \textbf{Descrizione}   \\ \hline
    Dimissione   & \small{\texttt{fse:dischargeDiagnosis}}  &  11535-2 & \small{Descrive l'elenco
    delle diagnosi di dimissione, in ordine di rilevanza.}      \\ \hline 
    Motivo del ricovero   & \small{\texttt{fse:reasonForAdmission}}     &    46241-6   & \small{Descrive la causa principale che ha determinato il ricovero del paziente
    attraverso la diagnosi di accettazione.}      \\ \hline
    Decorso ospedaliero   & \texttt{fse:progression}     &    8648-8   & \small{Descrivere l’andamento del ricovero, il percorso diagnostico,
    terapeutico, riabilitativo o assistenziale.}      \\ \hline
    \end{tabularx}
    \caption{Sezioni obbligatorie del documento LDO}
    \label{table:ldoSections}
\end{table}
%
\subsubsection*{DE (Documento di Esenzione)}
Descrive la tipologia di esenzioni posseduta dall’assistito che può essere
totale o parziale (Tabella \ref*{table:deSections}). Definito dalla classe
\texttt{fse:exemptionReport}, è associato al codice LOINC \texttt{"57827-8"}.
\begin{table}
    \begin{tabularx}{\textwidth}{|l|l|l|X|}
    \hline
    \textbf{Sezione}    & \textbf{Classe}  & \textbf{LOINC} & \textbf{Descrizione}   \\ \hline
        Esenzione & \texttt{fse:exemption}  & 57827-8 & \small{Contiene le
        informazioni sulle esenzioni.}      \\ \hline 
    \end{tabularx}
    \caption{Sezioni obbligatorie del documento DE}
    \label{table:deSections}
\end{table}
%
\subsubsection*{RSA (Referto di Specialistica Ambulatoriale)} 
RSA è finalizzato a definire uno standard per la refertazione di prestazioni
ambulatoriali specialistiche (visite mediche ed esami strumentali) che non
ricadano nella sfera della medicina di laboratorio, della radiologia ed imaging,
dell'anatomia patologica (Tabella \ref*{table:rsaSections}). Modellato tramite
\texttt{fse:outpatientSpecialistReport} e identificato dal codice LOINC
\texttt{"11488-4"}.
\begin{table}
    \begin{tabularx}{\textwidth}{|l|l|l|X|}
    \hline
    \textbf{Sezione}    & \textbf{Classe}  & \textbf{LOINC} & \textbf{Descrizione}   \\ \hline
    Prestazioni  & \texttt{fse:services}  & 62387-6 & \small{Riporta le prestazioni
    amministrative erogate, e, ove applicabile, le procedure operative e
    cliniche eseguite (es. somministrazione farmaci).}      \\ \hline 
    Referto  & \texttt{fse:osReport}  & 47045-0 & \small{Descrizione delle
    valutazioni del medico e dell’esito della prestazione.}      \\ \hline
    \end{tabularx}
    \caption{Sezioni obbligatorie del documento RSA}
    \label{table:rsaSections}
\end{table}
%
\subsubsection*{VPS (Verbale di Pronto Soccorso)}
VPS riassume i risultati di tutte le indagini eseguite in regime di urgenza in
Pronto Soccorso, attestando quanto effettuato per l’inquadramento diagnostico e
terapeutico (Tabella \ref*{table:vpsSections}). Rappresentato da
\texttt{fse:firstAidReport} e associato al codice LOINC \texttt{"59258-4"}.
\begin{table}
    \begin{tabularx}{\textwidth}{|l|l|l|X|}
    \hline
    \textbf{Sezione}    & \textbf{Classe}  & \textbf{LOINC} & \textbf{Descrizione}   \\ \hline
    Motivo della visita  & \texttt{fse:reasonForVisit}  & 46239-0 & \small{Descrive il motivo per cui il paziente accede al Pronto Soccorso, ed il
    problema, il sintomo principale riscontrato o percepito dal paziente.}      \\ \hline 
    Dimissione  & \texttt{fse:discharge}  & 28574-2 & \small{Descrive i dati
    relativi alla fase di dimissione, tra cui la diagnosi di dimissione, la
    prognosi, l’esito del trattamento.}      \\ \hline
    Triage & \texttt{fse:triage}  & 54094-8 & \small{Definisce la fase di triage
    dell’accesso in Pronto Soccorso.}      \\ \hline 
    Modalità di trasporto  & \texttt{fse:modeOfTransport}  & 11459-5 & \small{Specifica la modalità di trasporto (modalità arrivo) del paziente al
    Pronto Soccorso ed il responsabile dell’invio al Pronto Soccorso.}      \\ \hline
    \end{tabularx}
    \caption{Sezioni obbligatorie del documento VPS}
    \label{table:vpsSections}
\end{table}
    %
\subsubsection*{VAC (Vaccinazioni)}
VAC è il documento che attesta l’avvenuta somministrazione della/le
vaccinazione/i somministrate all’assistito in una certa data (Tabella
\ref*{table:vacSections}). Definito tramite \texttt{fse:immunization} e
identificato da codice LOINC \texttt{"87273-9"}.
\begin{table}
    \begin{tabularx}{\textwidth}{|l|l|l|X|}
    \hline
    \textbf{Sezione}    & \textbf{Classe}  & \textbf{LOINC} & \textbf{Descrizione}   \\ \hline
    \small{Scheda della singola Vacc.}  & \texttt{fse:immunizationCard}  & 11369-6 & \small{Descrive i dettagli della somministrazione}      \\ \hline 
    \end{tabularx}
    \caption{Sezioni obbligatorie del documento VAC}
    \label{table:vacSections}
\end{table}
%
\subsubsection*{RML (Referto di Medicina di Laboratorio)}
RML rappresenta il risultato di una prestazione svolta da uno specifico
laboratorio interno ad una struttura sanitaria. Rappresentato da
\texttt{fse:laboratoryMedicineReport} con codice LOINC \texttt{"11502-2"}.
L'identificativo LOINC della sezione (Tabella \ref*{table:rmlSections}) cambia
in base alla tipologia di esame.
\begin{table}
    \begin{tabularx}{\textwidth}{|l|l|c|X|}
    \hline
    \textbf{Sezione}    & \textbf{Classe}  & \textbf{LOINC} & \textbf{Descrizione}   \\ \hline
    Specialità  & \texttt{fse:specialty}  & * & \small{contiene i risultati di esami di laboratorio afferenti ad una singola specialità (ad esempio uno studio microbiologico) o a più specialità}      \\ \hline 
    \end{tabularx}
    \caption{Sezioni obbligatorie del documento RML}
    \label{table:rmlSections}
\end{table}
% 
\subsubsection*{PSS (Profilo Sanitario Sintetico)}: è il documento
socio-sanitario informatico redatto e aggiornato dal MMG/PLS, che riassume la
storia clinica dell’assistito e la sua situazione corrente conosciuta (Tabella
\ref*{table:pssSections}). In particolare, lo scopo del documento è quello di
favorire la continuità di cura. Definito dalla classe
\texttt{fse:summaryHealthProfile} con codice LOINC \texttt{"60591-5"}.
\begin{table}
    \begin{tabularx}{\textwidth}{|l|l|l|X|}
    \hline
    \textbf{Sezione}    & \textbf{Classe}  & \textbf{LOINC} & \textbf{Descrizione}   \\ \hline
    Allergie  & \texttt{fse:alerts}  & 48765-2 & \small{Raccoglie ogni
    informazione relativa ad allergie, reazioni avverse, ed allarmi passati
    o presenti inerenti il paziente, se ritenute rilevanti.}      \\ \hline 
    Terapie  & \texttt{fse:medications}  & 10160-0 & \small{Deputata alla
    registrazione di tutte le informazioni inerenti le terapie
    farmacologiche (prescrizioni, somministrazioni, ecc.)}      \\ \hline
    Lista dei Problemi	  & \small{\texttt{fse:HistoryOfDiseases}}  & 11450-4 & \small{Documenta tutti i problemi clinici rilevanti noti al momento in
        cui è stato generato il documento (problemi clinici, condizioni,
        sospetti diagnostici e diagnosi certe, sintomi attuali o passati).}      \\ \hline 
    Gravidanze e parto  & \small{\texttt{fse:HistoryOfPregnancies}}  & 10162-6 &
        \small{Include tutte le informazioni inerenti gravidanze (incluso aborti spontanei), parti, eventuali
        complicanze derivate e stato mestruale}
        \\ \hline
    Vaccinazioni  & \small{\texttt{fse:immunizations}}  & 11369-6 &
    \small{Riporta le informazioni relative allo stato attuale di
    immunizzazione (vaccinazioni) del paziente e alle vaccinazioni
    effettuate}      \\ \hline
    \end{tabularx}
    \caption{Sezioni obbligatorie del documento PSS}
    \label{table:pssSections}
\end{table}
%
\subsection{Certificato Vaccinale}
Attraverso la sezione \texttt{fse:immunizationCard} sarà possibile ripercorrere
le vaccinazioni effettuate negli anni, con conseguente produzione del
Certificato Vaccinale. Si rende necessaria l'introduzione dell'entità
\texttt{fse:substanceAdministration} (Figura \ref*{fig:immunizations}) al fine
di dettagliare l'atto di somministrazione mediante:
\begin{itemize}
    \item \texttt{effectiveTime}: istante di avvenuta vaccinazione;
    \item \texttt{routeCode}: via di somministrazione;
    \item \texttt{approachSiteCode}: parte anatomica interessata dalla somministrazione;
    \item \texttt{doseQuantity}: dose del farmaco inoculato.
\end{itemize}
Da questa si propagano varie relazioni:
\begin{itemize}
    \item \texttt{fse:prevents}: specifica la malattia che è in grado di prevenire;
    \item \texttt{fse:consume}: indica il farmaco impiegato;
    \item \texttt{fse:participant}: definisce la struttura sanitaria in cui si è
    svolta la somministrazione.
    \item \texttt{fse:booster}: tiene traccia degli eventuali richiami,
    identificati da un codice LOINC.
\end{itemize}

\begin{figure}
    \includegraphics[width=\textwidth]{images/immunizations.png}
    \caption[]{Estensione dell'entità \texttt{fse:immunization} (referto VAC).}
    \label{fig:immunizations}
\end{figure}
%
\subsection{PSS}
L'introduzione della relazione \texttt{fse:hasSection}, nel caso del
\texttt{summaryHealthProfile}, abilita alla raccolta di nuove informazioni
raggruppate per competenza:
\begin{itemize}
    \item \texttt{fse:alters}: sintetizza tutti i dati relativi ad eventuali
    allergie del paziente. Tramite \texttt{fse:hasAgent} si definisce l'agente
    che ha provocato l'allergia, mentre con \texttt{fse:adverseReaction} è
    possibile specificare le reazioni avverse. Infine con
    \texttt{fse:hasAllergy} si referenzia la descrizione completa dell'allergia
    di cui il paziente è affetto.
    \item \texttt{fse:historyOfDiseases}: storico delle eventuali patologia che
    affliggono il paziente. 
    \item \texttt{fse:medications}: la sezione raccoglie, tramite apposita
    relazione (\texttt{fse:hasTherapy}), tutte le terapie assegnate, e di ognuna
    anche la sostanza somministrata.
    \item  \texttt{fse:historyOfPregnancies}: corrisponde all'elenco delle
    gravidanze associate tramite la relazione \texttt{fse:hasPregnancy}, la
    quale include l'esito finale e il periodo.
\end{itemize}
\begin{figure}
    \includegraphics[width=\textwidth]{images/pss.png}
    \caption[]{Estensione dell'entità \texttt{fse:summaryHealthProfile} (referto PSS).}
    \label{fig:pps}
\end{figure}

\subsection{VPS e RSA}
Seguendo la logica delle precedenti estensioni anche per i referti VPS e RSA, vi
sono una serie di nuove relazioni dettate dall'integrazione di specifiche
sezioni (Figura \ref*{fig:vps-rsa}). Le sezioni dei due documenti hanno un
concetto comune, e cioè quello di \texttt{operatingProcedure}, il quale può
presentarsi in varie forme:
\begin{itemize}
    \item \texttt{fse:procedure}: indica procedure diagnostiche invasive,
    interventistiche, chirurgiche, terapeutiche non farmacologiche;
    \item \texttt{fse:substanceAdministration}: indica procedure di
    somministrazioni farmaceutiche come terapie, vaccinazioni e sedazioni;
    \item \texttt{fse:observation}: indica osservazioni eseguite e parametri
    clinici rilevati durante la prestazione;    
\end{itemize}
\begin{figure}
    \includegraphics[width=\textwidth]{images/vps-rsa.png}
    \caption[]{Estensione dei documenti VPS e RSA.}
    \label{fig:vps-rsa}
\end{figure}

\subsection{RML}
Attraverso l'impiego della sezione \texttt{fse:specialty}, il documento RML è in
grado di raccogliere gli esami effettuati (es. studio microbiologico) ed i
relativi risultati grazie alla relazione \texttt{fse:hasResult} (Figura
\ref*{fig:rml}). Infine è possibile ripercorrere la catena di esami eseguiti
mediante la relazione \texttt{fse:hasNext}.
\begin{figure}[]
    \centering
    \includegraphics[width=0.4\textwidth]{images/rml.png}
    \caption[]{Estensione del documento RML.}
    \label{fig:rml}
\end{figure}

\subsection{Prescrizioni}
\begin{itemize}
    \item \textit{Prescrizione farmaceutica} (\texttt{fse:pharma}): documento normato che riporta una
    previsione clinica e un ordine amministrativo, indirizzato alla struttura
    erogante (ex: FARMACIA), di fornire al paziente un insieme di presidi
    medicinali.
    \item \textit{Prescrizione specialistica} (\texttt{fse:specialist}):
    \item \textit{Prescrizione ricovero} (\texttt{fse:admission}): 
    \item \textit{Richiesta di trasporto} (\texttt{fse:transportRequest}): 
\end{itemize}
\begin{figure}[]
    \centering
    \includegraphics[width=\textwidth]{images/prescription.png}
    \caption[]{Schema dell'entità \texttt{fse:prescription}.}
    \label{fig:prescription}
\end{figure}

\newpage
\nocite{*} % Includes all references from the `references.bib` file
\bibliographystyle{plain}
\bibliography{references}

\end{document}